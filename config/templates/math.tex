\usepackage{amssymb, amsfonts, amsbsy, amsmath, amsthm, amscd}

% Graphik und kommutative Diagramme
\usepackage{graphicx}
\usepackage{tikz}
\usepackage[all,cmtip]{xy}
\usepackage{faktor}

% Gruppen, Ringe und Körper
\newcommand{\NN}{{\mathbb N}}
\newcommand{\ZZ}{{\mathbb Z}}
\newcommand{\QQ}{{\mathbb Q}}
\newcommand{\RR}{{\mathbb R}}
\newcommand{\CC}{{\mathbb C}}
\newcommand{\FF}{{\mathbb F}}
\newcommand{\Gal}{\operatorname{Gal}}

% Potenzmenge
\renewcommand{\P}{{\mathfrak P}}

% Abbildungen
\newcommand{\id}{\operatorname{id}}
\newcommand{\Abb}{\operatorname{Abb}}
%\newcommand{\ker}{\operatorname{ker}}
\newcommand{\coker}{\operatorname{coker}}
\newcommand{\Hom}{\operatorname{Hom}}
\newcommand{\Aut}{\operatorname{Aut}}
\newcommand{\sheaves}{\operatorname{Sheaves}}
\newcommand{\tor}{\operatorname{Tor}}
\newcommand{\cone}{\operatorname{cone}}
\newcommand{\im}{\operatorname{Im}}
\newcommand{\coim}{\operatorname{Coim}}
\newcommand{\Ob}{\operatorname{Ob}}
\newcommand{\cardinality}[1]{\vert #1 \vert}
\newcommand{\ord}{\operatorname{ord}}
\newcommand{\ideal}[1]{(#1)}
\newcommand{\characteristic}{\operatorname{char}}
\newcommand{\LC}{\operatorname{LC}}

% Kategorien
\newcommand{\cat}[1]{\mathcal{#1}}
\newcommand{\ccomplex}[2]{\operatorname{\textbf{Ch}}#1{\geq0}(\mathcal{#2})}
\newcommand{\ch}{\textrm{\textup{\textbf{Ch}}}}
\newcommand{\ab}{\textrm{\textup{\textbf{Ab}}}}
\newcommand{\Rmod}{R\textrm{\textup{\textbf{-mod}}}}
\newcommand{\Amod}{A\textrm{\textup{\textbf{-mod}}}}
\newcommand{\Agr}{A\textrm{\textup{\textbf{-gr}}}}
\newcommand{\Kom}[1]{\textrm{\textup{\textbf{Kom}}}(\cat{#1})}

% Sonstige
\newcommand{\setvert}{\;\vert\;}
\newcommand{\divides}{\,\vert\,}


% Theorem Definitionen mit durchlaufender Numerierung
%\swapnumbers
\theoremstyle{definition}
\newtheorem{theorem}[subsection]{Theorem}
\newtheorem{lemma}[subsection]{Lemma}
\newtheorem{corollary}[subsection]{Corollary}
\newtheorem{proposition}[subsection]{Proposition}
\newtheorem{algorithm}[subsection]{Algorithm}
\newtheorem{exercise}[subsection]{Exercise}
\newtheorem{porism}[subsection]{Porism}
\newtheorem{criterion}[subsection]{Criterion}
\newtheorem{construction}[subsection]{Construction}

\newtheorem{definition}[subsection]{Definition}
\newtheorem{assumption}[subsection]{Assumption}
\newtheorem{axiom}[subsection]{Axiom}
\newtheorem{notation}[subsection]{Notation}

\newtheorem{aufgabe}[subsection]{Aufgabe}

\theoremstyle{plain}
\newtheorem{remark}[subsection]{Remark}
\newtheorem{example}[subsection]{Example}
\newtheorem{note}[subsection]{Note}

\newcommand{\comment}[1]{\textcolor{red}{#1}}

\DeclareRobustCommand\longtwoheadrightarrow
     {\relbar\joinrel\twoheadrightarrow}

% Restriction of map
\newcommand\restr[2]{{% we make the whole thing an ordinary symbol
  \left.\kern-\nulldelimiterspace % automatically resize the bar with \right
  #1 % the function
  \littletaller % pretend it's a little taller at normal size
  \right|_{#2} % this is the delimiter
  }}

\newcommand{\littletaller}{\mathchoice{\vphantom{\big|}}{}{}{}}