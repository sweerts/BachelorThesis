\usepackage[utf8]{inputenc}

\usepackage[english]{babel}
\usepackage{enumitem}

% Etwas Farbe, man kann sogar per rgb Wert definieren
\usepackage{color}
\newcommand{\black}{\color{black}}
\newcommand{\red}{\color{red}}
\newcommand{\green}{\color{dgreen}}
\newcommand{\blue}{\color{blue}}
\newcommand{\magenta}{\color{magenta}}

\definecolor{petrol}{RGB}{0,110,79}

\definecolor{darkblue}{rgb}{0,0,.5}

% Das macht, dass man im pdf auf die Verweise klicken kann
\usepackage[colorlinks,linkcolor=,citecolor=,urlcolor=petrol]{hyperref}

\setlength{\parindent}{0pt}
\setlength{\parskip}{5pt}

\newcommand{\TODO}[1]{\textcolor{red}{\Large \bfseries #1}}

\usepackage[
    backend=biber,
    style=alphabetic,
    sorting=ynt
]{biblatex}
\addbibresource{bibliography.bib}